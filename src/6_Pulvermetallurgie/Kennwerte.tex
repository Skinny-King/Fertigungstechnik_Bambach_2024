\begin{tiny}
    \begin{align*}
    d_{50} &= \text{Medianwert, 50\% der Partikel sind kleiner als dieser Wert} \\
    d_{90} &= \text{90\% der Partikel sind kleiner als dieser Wert} \\
    d_{10} &= \text{10\% der Partikel sind kleiner als dieser Wert} \\
    | d_{90} - d_{10} | &= \text{Spannweite}
\end{align*}
\end{tiny}


\textbf{Mastersinterkurve:}
rel. Dichte $\rho (t,T)$  vs log. Sinterarbeit $\log \Theta $\\
Unabhängig der Heizrate folgen alle Kurven dem gleichen Trend\\

\textbf{Annahmen}: Es existiert nur ein einzelner Diffusionsmechanismus, 
Diffusion thermisch aktiviert, Korngröße/Mikrostruktur variiert 
nur mit der Dichte, Oberflächendiffusion vernachlässigt.\\

\textbf{Einschränkungen}: $Q$ (Aufheizrate) hängt von der Partikelgrössen-\\
verteilung (PSD) und der Chemie ab. MSC ist empfindlich für 
Änderungen in der Materialzusammensetzung und Prozessbedingungen.\\
Bei $\rho (t,T) > 90 \%$ nimmt $Q$ ab.\\