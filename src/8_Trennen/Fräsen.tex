Beim Fräsen wird die notwenidige Relativbewegung zwischen Werkzeug und Werkstück durch eine kreisförmige Schnittbewegung des Werkzeugs und eine Vorschubachse erzielt. Die Schneide ist nicht ständg im Eingriff und unterliegt daher thermischen und mechanischen Wechselbelastungen.\\

\textbf{Hochgeschwindigkeitszerspanung (HSC):}\\
Geringe Schnitttiefe, hohe Zerspanleistung hohe Schnittgeschiwndigkeit, hohe Vorschubgeschiwndigkeit. Bedingungen sind kurze kurze Werzeugeingriffszeiten und hohe Eingriffsfrequenzen. Vorteile sind hohe Zeitspanvolumina, geringere Gratbildung, verminderte Randzonenbeeinflussung, gesteigerte Massgenauigkeit, Bearbeitung dünnwandiger Teile, kaum Schiwndungen.\\