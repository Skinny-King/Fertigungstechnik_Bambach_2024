\textbf{Messen und Prüfen:}
\begin{itemize}
    \item Messen: Quantitative und vergleichende Erfassung einer Eigenschaft
    \\Ermitteln einer Länge/ Winkel mit einem Messgerät $\rightarrow$ Messwert
    \item Prüfen: Qualitative Beurteilung einer gemessenen Eigenschaft
    \\Prüfen ob Gegenstand die geforderten Merkmale aufweist $\rightarrow$ Funktionstüchtig
    \item Kalibrieren: Vergleich eines Messwertes mit dem genormten Referenzstandart,
    Dokumentieren der Abweichung, Berechnung der Messunsicherheit und Erstellen des Zertifikates.
\end{itemize}
\hspace{0.05\linewidth}

\textbf{Berührende Messverfahren:}\\
Geometrie der Tastspitze beeinflusst 
gemessene Rauheit. Empfindliche Oberflächen können durch die Berührung der Tastspitze 
beschädigt werden. Flächige Messung nur mit hohem Aufwand 
(Messzeit) realisierbar.\\

\textbf{Optische Messverfahren:}\\
Hohe Messgeschwindigkeit. Schwierigkeiten bei spiegelnden 
Oberflächen. Oberfläche muss sauber sein\\

\textbf{Messunsicherheiten} entstehen durch: Vibrationen, elektromagnetische 
Felder, Staub und Temperaturschwankungen.

