\textbf{Key Performance Dimensions:} (möglichst minimieren); Vorlaufzeit, Qualität, Emissionen, Kosten \\
\textbf{Concurrent engineering:} Ermöglicht gleichzeitige Arbeitsschritte (nicht sequenziell/ nicht linear). \\
\textbf{CAE:} Computer Aided Engineering \\
\textbf{CAM:} Computer Aided Manufacturing \\
\textbf{Lean Manufacturing:} Kosten minimieren und Produktivität maximieren. Fünf Prinzipien: Definition des Wertes für Kunden, Identifikation des Wertstroms, Umsetzung des FLussprinzips, Einführung des Pull-Prinzips, Streben nach Perfektion. \\
\textbf{Verschwendung:} Überproduktion, Unnötige Bewegung, Hohe Bestände, Transport, Wartezeiten, Ineffizienz, Nacharbeit/Ausschuss.\\
\textbf{Kanban:} "Visuelle Karte". Klassisches Pull-System. Selbstregulierende Regelkreise gewährleiten Materialversorgung. Es wird alles nur auf Nachfrage ausgeführt. Verhindern von Überproduktion. \\
\textbf{Automatisierungspyramide:} Feldebene (Sensoren \& Aktoren), Steuerungsebene (Steuerungen und Überwachungsanlagen), Leitebene (Erfasst und Daten und zeigt diese an), Unternehmensebene (Regelt und steuert die ganzen Systeme).\\