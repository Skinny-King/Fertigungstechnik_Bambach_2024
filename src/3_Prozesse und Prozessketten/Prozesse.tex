\textbf{Planungsprozesse:}\\
\textbf{Key Performance Dimensions:} Vorlaufzeit$\downarrow$ , Qualität $\uparrow $, Emissionen $\downarrow $, Kosten $\downarrow $\\
\textbf{Concurrent engineering:} Ermöglicht gleichzeitige Arbeitsschritte (nicht seq./ nicht lin.). \\
\textbf{CAE:} Computer Aided Engineering \\
\textbf{CAM:} Computer Aided Manufacturing \\
\textbf{Lean Manufacturing:} Kosten $\downarrow$ und Produktivität $\uparrow $\\
Fünf Prinzipien: Definition des Wertes für Kunden, Identifikation des Wertstroms, \\Umsetzung des Flussprinzips, Einführung des Pull-Prinzips, Streben nach Perfektion. \\
\textbf{Verschwendung:} Überproduktion, Unnötige Bewegung, Hohe Bestände, Transport, Wartezeiten, Ineffizienz, Nacharbeit/Ausschuss.\\
\textbf{Kanban:} "Visuelle Karte". Klassisches Pull-System. Selbstregulierende Regelkreise gewährleiten Materialversorgung. Es wird alles nur auf Nachfrage ausgeführt. Verhindern von Überproduktion. \\
\textbf{Automatisierungspyramide:} Feldebene (Sensoren \& Aktoren), Steuerungsebene (Steuerungen und Überwachungsanlagen), Leitebene (Erfasst und Daten und zeigt diese an), Unternehmensebene (Regelt und steuert die ganzen Systeme).\\


\textbf{Prozesse:}
\begin{itemize}
    \item Urformen  (Gießen, Sintern, 3D)
    \item Umformen  (Schmieden, Walzen, Tiefziehen)
    \item Trennen (Drehen, Fräsen, Bohren, Scherschneiden)
    \item Fügen (Schweissen, Löten, Kleben)
    \item Beschichten  (Galvanisieren, Lackieren, Aufdampfen)
    \item Stoffeigenschaften ändern  (Härten, Glühen, Magnetisieren)
\end{itemize}
\hspace{0.05\linewidth}

