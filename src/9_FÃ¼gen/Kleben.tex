\textbf{Definition:} \\
Beim Kleben werden gleiche oder unterschieldiche 
Stoffe durch eine aushärtende Zwischenschicht stoffschlüssig 
und nicht lösbar verbunden. Die Klebewirkung beruht auf der 
\textbf{Adhäsionskraft des Klebstoffes an den Fügeflächen} und der 
\textbf{Kohäsionskraft im Innern der Klebeschicht}. Klestoffe können 
gut Scherkräfte aufnehmen.\\

\textbf{Vorteile:}\\
Keine Gefügeänderung, gleichmässige Spannungsverteilung, 
viele Werkstoffkombinationen, dichte Verbindungen, 
wenig Passarbeit nötig, grossflächige Verbindungen.\\

\textbf{Nachteile:}\\
grosse Fügeflächen nötig, geringe Festigkeit, geringe Temperaturbeständigkeit, 
lange und komplizierte Aushärtung, keine 
zerstörungsfreie Prüfung möglich, anfällig für Schälkräfte.\\

\textbf{Physikalisch abbindende Klebstoffe:}
\begin{itemize}
    \item Lösungsmittelklebstoffe härten durch Ablüften des 
    Lösungsmittels. Sie basieren auf gelösten Kautschuken.
    \item Schmelzklebstoffe härten durch Abkühlung
    \item Dispersionsklebstoffe benötigen. Wärmeeinwirkung zum 
    Aushärten. Sie basieren auf PVC, Weichmachern, Füllstoffen 
    und Haftvermittlern.
\end{itemize}

\textbf{Chemisch abbindende Klebstoffe:}
\begin{itemize}
    \item Polmerisationsklebstoffe werden katalytisch durch 
    Feuchtigkeit ausgelöst.
    \item Bei Polyadditionsklebstoffen reagieren mindestens zwei 
    unterschiedliche Stoffe miteinander verbunden.
    \item Polykondensationsklebstoffe reagieren unter der Abspaltung 
    von flüchtigen Stoffen. Benötigen Pressdruck von mindestens
     $40 N/cm$ und oft erhöhte Temperaturen.
\end{itemize}